\documentclass{standalone}
\usepackage{tikz}
\usepackage{pgfplots}
\begin{document}
\pgfmathdeclarefunction{myfunct}{1}{\pgfmathparse{#1^2}}
\pgfmathdeclarefunction{diff}{1}{\pgfmathparse{#1-0.25}}
    \begin{tikzpicture}[
            >=stealth, %% arrow tips
        ]
        \begin{axis}[
                blue,
                axis x line=middle,
                axis y line=center,
                every axis x label/.style={at={(current axis.right of origin)},anchor=north},
                every axis y label/.style={at={(current axis.above origin)},anchor=east},
                xmin=-0.5,xmax=1.5,
                ymin=-0.5,ymax=3,
                xtick=\empty,
                ytick=\empty,
                xlabel={$x$},
                ylabel={$y$},
            ]

            %% draw the plot:
            \addplot [black,samples=100] {myfunct(x)};
            \addplot [red,samples=100] {diff(x)};

            %% define some coordinates that we need later:
            \def\xb{0.5}
            \pgfmathsetmacro{\yb}{myfunct(\xb)}
            \path (axis cs:\xb, \yb) coordinate (0);

            \path (axis cs:0, 0) coordinate (origin);
        \end{axis}

        %% draw the black lines:
        \tikzset{marker/.style={shorten <=-3pt,shorten >=-3pt}} %% expand the lines
        \draw [marker] (origin-|0) -- (0);
        \draw [marker] (origin|-0) -- (0);

        \path (origin) ++(10pt,10pt) coordinate (offset);

        \node at (origin-|0) [below,yshift=-3pt,red!50!blue] {$x$};
        \node at (0-|origin) [left,xshift=-3pt,black!50!blue] {$y$};
    \end{tikzpicture}
\end{document}